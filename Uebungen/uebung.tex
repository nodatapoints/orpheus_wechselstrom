\documentclass[a4paper]{scrartcl}                        % Blatt
\usepackage{amsmath,amsthm,amssymb,mathtools}
\usepackage{listings}
%\usepackage[ngerman]{babel}
\usepackage{graphicx}
\usepackage{subcaption}
\usepackage{harpoon}
\usepackage[ngerman]{babel} 

\newcommand{\vect}[1]{\overrightharp{\ensuremath{#1}}}
\newcommand{\field}[1]{\par\begin{large}{\vspace*{0.5cm}\noindent{}\textbf{#1}\vspace*{-1mm}}\end{large}}
\newcommand{\rom}[1]{\uppercase\expandafter{\romannumeral #1\relax}}
%\newcommand{\problem}[2]{{\par\noindent{}\textit{Problem {\uppercase\expandafter{\romannumeral #1\relax}}.} #2}}
\newenvironment{problem}[3]{{\par\vspace*{4.5mm}\noindent{}\textit{Aufgabe {\uppercase\expandafter{\romannumeral #1\relax}}:} #2}\vspace*{+0.2cm}\hfill{\textit{#3}}\\\hspace*{0.8cm}\hfill\begin{minipage}{\dimexpr\textwidth-1.3cm}}{\end{minipage}\hspace*{0.5cm}}
% don't fucking ask! (im 4h into this and it __works__)
% \vspace*{+0.2cm} -> \vspace*{-0.2cm} wenn die erste Zeile nen align is.

\begin{document}

\title{Übungsaufgaben komplexe Wechselstromrechnung}
\author{Kamal Abdellatif, Philip Geißler}
 
\maketitle

\field{Impendanz-, Strom, Spannungs-, Leistungs- und Arbeitsrechnung}
\begin{problem}{1}{Kenngrößen des kompensierten $RC$-Spannungsteilers}{placeholder}
hallo
\end{problem}


\field{Konzeption komplexer Wechselstrom}
\begin{problem}{3}{Effektive und Genaue verrichtete Arbeit}{placeholder}
hallo hallo
\end{problem}

\end{document}
