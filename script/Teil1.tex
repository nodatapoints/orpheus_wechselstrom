\section{Wechselstromkreise}
\subsection{Grundlagen}
Im folgenden werden wir uns mit Wechselstromkreisen aus idealen \emph{passiven Bauelementen} beschäftigen, welche Widerstand, Spule und Kondensator sind. Diese Elemente werden als \emph{passiv} bezeichnet, da sie kein Signal verstärken können.

Im Gegensatz zu Gleichstromkreisen beschränken wir uns auf Spannungs- und Stromquellen, welche sinusförmige Signale über die Zeit der Form
\begin{equation}\label{eq:sinusform}
    U(t) = U_0\sin(\omega t + \phi_0) \qquad
    I(t) = I_0\sin(\omega t + \phi_0)
\end{equation}
erzeugen. Hier ist $U_0, I_0$ die \emph{Amplitude}, $\phi_0$ die \emph{Phase} und  $\omega$ die sogenannte \emph{Kreisfrequenz}. Die Periodendauer $T$ ergibt sich aus der Frequenz $f = \omega/\tau$ als $T = 1/f = \tau/\omega$.
Im folgenden wird die Kreisfrequenz $\omega$ beliebig aber fest sein.

Diese Einschränkungen scheinen zunächst sehr limitiert, doch es wird sich durch diese Annahmen offenbaren, wie sich die scheinbar verschiedenen Bauteile sich mathematisch vereinen lassen. Dieses schöne und anschauliche Modell legt die Grundlage für viele komplexere Themen der Elektronik. Wir hoffen mit unserem Kurs einen Einblick und vor allem eine Intuition für Wechselstromkreise zu schaffen, welche die unterliegende Struktur offenbart.

\subsection{Die Bauteile}
\paragraph*{Der Widerstand}

