\documentclass[margin=5mm]{standalone}
\usepackage{siunitx}
\usepackage[EFvoltages, europeanresistors]{circuitikz}
%\usepackage[symbols]{circuitikz}
\usepackage{tikz}
\begin{document}

% % <------ normale line width
%\begin{circuitikz}[>=latex, 
%thin, 
%common/.style={fill=white}
%]
%  \draw
%  (0,0) node[] (UL) {} % <------
%  to[short, o-*] (3,0) 
%  to[C, l={C}, *-*] (3,2)
%  (3,0) to[short, *-o] (4,0) node[] (UR){} % <------
%  (0,2) node[] (OL) {} % <------
%  to[short,o-](0.4,2) 
%  to[R, l_={R}, -](1.5,2)  % <---
%  to[L, l_={L},-*](3,2)      % <--
%  (3,2) to[short, *-o] (4,2) node[] (OR){}; % <------
%
%\tikzset{every path/.style={very thin}}
%\draw[<->] (UL) -- (OL) node[midway]{p$_\textsf{in}$};
%\draw[<->] (UR) -- (OR) node[midway]{p$_\textsf{out}$};
%\draw[->, transform canvas={yshift=5mm}] (OL.center) -- +(7mm, 0) node[ anchor=west]{q$_\textsf{in}$};
%\draw[<-, transform canvas={yshift=5mm}] (OR.center) -- +(-7mm, 0) node[ anchor=east]{p$_\textsf{out}$};
%\end{circuitikz}

\ctikzset{bipoles/thickness=1}
% \begin{circuitikz}[>=latex]
%   \draw
%   (-1.5,0) to[sV, v=$U$, i=$I$] ++(2,0) -- ++(0,0.5) 
%   to[R, -, name=R1] ++(1.5,0) -- ++(0,-0.5) -- ++(0.5,0) -- ++(0,-1) -- ++(-4,0) -- ++(0,1);
%   \draw
%   (0.5,0) to[short, *-] ++(0,-0.5) to[R, -, name=R2] ++(1.5,0) to[short, -*] ++(0,0.5);
%   \node  at (R1.center) {$X_1$};
%   \node  at (R2.center) {$X_2$};
% \end{circuitikz}

\begin{circuitikz}[>=latex]
  \draw
  (-1.5,0) to[sV, v=$U$, i=$I$] ++(2,0) 
  to[R, -, name=R1] ++(1.5,0) to[R, -, name=R2] ++(1.5,0) -- ++(0,-1) -- ++(-5,0) -- ++(0,1);
  \node  at (R1.center) {$X_1$};
  \node  at (R2.center) {$X_2$};
\end{circuitikz}


\end{document}
